\begin{abstract}
Distributed constraint optimization allows to solve problems in domains like scheduling, traffic flow management or sensor network management. It is a well-researched field and various algorithms have been proposed. However, the dynamic nature of some of these problems in the real world have been overlooked by researchers and problems are often assumed to be static during the course of the computation. The benchmarking of distributed constraint optimization algorithms (DCOP) with changing problem definitions currently lacks a solid theoretical foundation and standardized protocols. This thesis aimed to measure the performance of different types of DCOP algorithms on dynamic problems with a focus on local-iterative algorithms and especially on the MaxSum algorithm and possibly contribute to the field. A complete, a local-iterative message-passing and a local-iterative approximate best-response algorithm for distributed constraint optimization have been implemented for comparison.  In the implementation of the MaxSum algorithm, a variation of the usual graph structure has been attempted.  As a real-world use case for benchmarking, the meeting scheduling problem has been mapped as distributed constraint optimization problem. A framework has been designed that allows dynamic changes to constraints, variables and the problem domain during run-time. The algorithms have been benchmarked in a static, as well as in a dynamic environment, with various parameters and with a focus on solution quality over time. This thesis further proposes a solution to store, further process and monitor the results of the computation in real-time without affecting the performance of the algorithms.
\end{abstract}
