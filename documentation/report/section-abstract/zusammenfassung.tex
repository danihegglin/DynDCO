\begin{zusammenfassung}
Distributed constraint optimization erm{\"o}glicht Probleml{\"o}sungen in beispielweise Terminplanung, Verkehrsflusskontrolle oder dem Management von Sensor Netzwerken. Es ist ein gut erforschtes Feld und es wurden viele verschieden Algorithmen zur Berechnung vorgestellt. Allerdings wird h{\"a}ufig von einer statischen Problemdefinition ausgegangen und der Aspekt von in der Realit{\"a}t h{\"a}ufig auftretenden {\"A}nderungen an der Problemstellung findet wenig Beachtung. Ausserdem fehlt es an einem theoretischen Fundament und standardisierten Verfahren um die Performanz von DCOP Algorithmen hinsichtlich sich {\"a}ndernder Probleme zu erfassen. Diese Arbeit hatte das Ziel das Verhalten und die Leistung von verschieden Arten von DCOP Algorithmen in dynamischen Umgebungen mit einem Fokus auf lokale, iterative Algorithmen und Hauptaugenmerk auf den MaxSum Algorithmus zu untersuchen. Zum Vergleich wurde eine komplette und eine lokal, iterative "message-passing" sowie eine "best-response" Variante implementiert. Zum Test eines realen Problems wurde Terminplanung ausgew{\"a}hlt und als DCOP formuliert. Es wurde ausserdem ein Framework entwickelt, welches die dynamische {\"A}nderungen von Constraints, Variablen und der Problemdom{\"a}ne erm{\"o}glicht. Die Algorithmen wurden mit Fokus auf Qualit{\"a}t {\"u}ber Zeit, sowohl in einer statischen wie auch in einer dynamischen Umgebung getestet. Diese Arbeit schl{\"a}gt ausserdem eine L{\"o}sung zur Speicherung, Weiterverarbeitung und {\"U}berwachung der Resultate der Berechnungen in Echtzeit vor, welche die Performanz der Algorithm nicht beeinflusst.
\end{zusammenfassung}


