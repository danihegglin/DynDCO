\chapter{Conclusions}
\label{c:conclusions} 
%---------------------------------------- what has been done & achieved
In this thesis, the meeting scheduling has been mapped to a distributed constraint optimization problem. The formal definition has been derived from the literature and a local utility function has been formulated. An general description was given on complete distributive, local-iterative best-response and local-iterative message passing algorithms in the research area of distributed constraint optimization based on the categorizations of \cite{Chapman2011}. Further, the specific algorithms MaxSum, Maximum-Gain-Message (MGM) and  Distributed Pseudotree Optimization Procedure (DPOP) have been described in terms of their graph structure and communication behaviour and advantages as well as disadvantages have been pointed out. The algorithms have been mapped to the programming paradigm of the Signal/Collect framework on top of an implemented framework for benchmarking and dynamic changes based on these descriptions and specified to the meeting scheduling problem. Solutions had to be found to map soft constraints as well as hard constraints into the graph structure and vertices for all three algorithms in a manner that the performance values could be compared. Additionally, a monitoring and storage solution has been proposed that allows for immediate processing of values from the graph and real-time monitoring of the performance.
\newline\newline
 In the mapping of MaxSum, an approach for the graph structure has been taken that varies from the commonly described factor graphs in the papers. During the mapping process, problems arose when only on factor node was present in the graph because of the inherent message structure defined by the algorithm formulation. Instead of binary connections to a factor node from to variable nodes, it was chosen to allow multiple respectively k-ary connections. This choice was based on the fact that factor graphs are based on bipartite graphs, which allow such connection setups.
 \newline \newline
The framework has proven to be a good starting point to benchmark dynamically changing problems during runtime and could be further extended for future research. The monitoring platform has proven to be very helpful in the process of implementation as well as during the evaluation. The benchmarking has delivered some interesting data on the performance of the algorithms. The comparison between the three approaches in terms of Time to Solution has shown the abilities of the local-iterative implementations to deliver a certain level of quality quicker than the complete variation, but also reach a lower median utility respectively do not converge every time. Surprisingly, the MGM algorithm did fairly well in asynchronous mode even if the implementation does not wait for a complete set of neighbour messages. The reason could be the limited amount of participants per meeting and the low amount of delay in the system. The MaxSum algorithm has shown an interesting property of scaling very well and even improving the convergence rate over the amount of agents in asynchronous mode, whereas it did not scale well in synchronous mode. The influence of problem density has shown to be comparable along the local-iterative algorithms. To benchmark the dynamic properties, a fairly new proposal by \cite{Maillera} has been adjusted to utilities instead of conflicts and tested. This evaluation method showed to deliver useful results. Further, another benchmark method has been attempted, which also could help to gain more insight into dynamic constraint optimization problems.

