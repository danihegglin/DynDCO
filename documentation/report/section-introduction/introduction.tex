\chapter{Introduction 2}

The goal of this thesis is explore dynamic distributed constraint optimization. Constraint optimization itself is well-researched field and even distributed constraint optimization has been under research. The aspect of having a dynamic environment and dynamic meaning changing constraints is very under-developed and underinvestigated.\newline
\newline
The thesis tries to bring further what previously has been achieved in several bachelor and master thesis. The main goal is to show the capabilities of the signal-collect framework and the performance in distributed environments. Important is also mapping of the chosen Meeting Scheduling Problem and And how algorithms like the Max-Sum algorithm can be extended to better perform in dynamic environments. All those implementations shall be tested in benchmarking situations.\newline
\newline
First I will give an overview about various definitions and aspects of constraint optimization in general, as well as the aspects of  distributed and dynamic environments. I am also going to give an overview about different approaches of algorithms to solve constraint optimization problems and their advantages and disadvantages in various contexts. The I will choose appropriate algorithms for the experiments and comparisons and map them to the meeting scheduling problem. I also will design an attempt of introducing dynamic environments that fits into the given framework provided by DDIS. Then I will describe the implemenation details of the algorithms and the testbed, as well as the  dynamic environment modelling. After that I will conduct experiments in various testsets, various algorithms, various setups and analyze the limitations of existing algorithms to the problem in dynamic environments, as well as options to optimize defined benchmarks like bounce-back time after amount of change, time to reach certain quality, etc.
