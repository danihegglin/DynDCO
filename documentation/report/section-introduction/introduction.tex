\chapter{Introduction}

\section{Motivation \& Goal}
Constraint optimization allows to solve problems in various areas. The distributed nature of  many of those problems has been extensively adressed by research in distributed constraint optimization and the formulation of numerous algorithms with diverse design approaches. However, most of those algorithms were designed and most studies are conducted on the premise that problems are static in their predefined state and do not change over the course of the problem solving process. This might work as as a step by step procedure, but it does not work in a distributed manner with multiple agents \cite{Petcu2007}. But as a matter of fact, many problems have dynamic properties and in a world with ever increasing complexity and speed, those become ever more relevant. Constraints can change, but also the involved variables as well as the problem domain itself. One could for instance imagine a real-time business analytics software continously calculating the most optimal solution to a problem or drones exploring an area to find survivors of a earthquake. Research in dynamic distributed constraint optimization is sparse. Mailler et al. attribute this to a lack of standardized benchmarks \cite{Mailler2014}. There has also been some research done at DDIS on constraint optimization problems with a special focus on max sum.
\newline \newline
This thesis tries to explore dynamic distributed constraint optimization by implementing three different algorithm approaches and compare their performance in a dynamic environment. The main goal here is to understand the behaviour of the max-sum algorithm that should be able to handle changing properties. It is further a goal to show ways of benchmarking these type of problems from various aspects. The example problem for the thesis will be meeting scheduling and the implementation will be carried out on signal/collect, the graph processing engine developed at  DDIS at ifi UZH. The goal here would also be to show the capabilities of the engine to handle these kind of problems. 

% should include the statement from the beginning

\section{Structure}
First, an overview will be given about various definitions and aspects of constraint optimization in general, as well as the aspects of  distributed and dynamic environments. Further, an overview will be provided about different approaches of algorithms to solve constraint optimization problems and their advantages and disadvantages in various contexts as well as the family they are coming from. 

In the design part, the problem definition and the mapping of the problem on to the three algorithms will be explained. Also, the common denominator parts of the algorithms and an interface for dynamic changes will be explained. The design of the data collection will also be briefly introduced.

In the implementation part, details of the mapping to signal/collect will be explained and a few words on the testbed solution will be added.

After that I will conduct benchmarks to measure first the properties of the algorithms related to the problem mapping. Second, I will run various tests on changing constraints, variables and the domain with different rates and different problem densities to determine the dynamic performance of the algorithms on the particular problem of meeting scheduling. Both benchmarks will be discussed, further work and limitations will be pointed out and a conclusion will be given.