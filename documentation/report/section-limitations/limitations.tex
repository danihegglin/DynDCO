\chapter{Limitations \& Future Work}
\label{c:limitations}
%----------------------------------- Limitations -----------------------------
Limitations inherent in this thesis are the number and types of algorithms and approaches, as well as the focus on the specific problem of meeting scheduling. To generalize the results of this thesis on the performance of the algorithms in respect to solution quality over time and in a dynamic constraint environments, one would need to benchmark different problems with other constraint settings and compare additional algorithms on the framework. 
%----------------------------------- Future Work -----------------------------
Future work could include the benchmarking of other problems than meeting scheduling with the given algorithm implementations and structure. Especially, the MaxSum algorithm seems promising for problems that require a quick solution to a problem like network traffic routing or high-frequency sensor networks. The meeting scheduling problem could also be further explored by increasing the amount of maximum meeting participations of an agent. The amount of participations was limited during the course of this thesis to not further expand the number of possible cases for benchmarking, but it could be interesting to see how this affects the overall performance of the algorithms as the complexity to find a converging solution increases with more participants in a meeting.
 \newline \newline
 Two aspects of dynamic problem changes have not been investigated in the benchmarking chapter of this thesis. The first aspect would be changing of domain spaces during run-time. It would be a possibility to study the effects of  increases and decreases at given intervals and percentages. Decreasing domain value sets supposedly render a complete solution impossible by for example in meeting scheduling reducing the timeslots to a minimum.  Increasing the domain space as the only dynamic property would be expected to have a low impact as the number of meetings does not increase and therefore no need for more timeslots would emerge. It would therefore be interesting to see how the combination of multiple change types affects the problem solution process as a second aspect. A dynamic environment like this could for example be a possibility for a real-time scheduling system, which continously integrates new information into a overall problem. One could investigate the effects of dynamically adding variables and constraints at the same time and simultanously increasing the domain space. In this case, the stability properties of the algorithms would be of great importance too and could also be tested and further studied.
 

 